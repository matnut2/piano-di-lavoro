%----------------------------------------------------------------------------------------
%   USEFUL COMMANDS
%----------------------------------------------------------------------------------------

\newcommand{\dipartimento}{Dipartimento di Matematica ``Tullio Levi-Civita''}

%----------------------------------------------------------------------------------------
% 	USER DATA
%----------------------------------------------------------------------------------------

% Data di approvazione del piano da parte del tutor interno; nel formato GG Mese AAAA
% compilare inserendo al posto di GG 2 cifre per il giorno, e al posto di 
% AAAA 4 cifre per l'anno
\newcommand{\dataApprovazione}{Data}

% Dati dello Studente
\newcommand{\nomeStudente}{Matteo}
\newcommand{\cognomeStudente}{Soldà}
\newcommand{\matricolaStudente}{1226319}
\newcommand{\emailStudente}{matteo.solda.1@studenti.unipd.it}
\newcommand{\telStudente}{+ 39 342 029 3341}

% Dati del Tutor Aziendale
\newcommand{\nomeTutorAziendale}{Prof. Mauro}
\newcommand{\cognomeTutorAziendale}{Conti}
\newcommand{\emailTutorAziendale}{conti@math.unipd.it}
\newcommand{\telTutorAziendale}{+39 049 827 1488}
\newcommand{\ruoloTutorAziendale}{Prof.}

% Dati dell'Azienda
\newcommand{\ragioneSocAzienda}{Università degli Studi di Padova}
\newcommand{\indirizzoAzienda}{Dipartimento di Matematica - Via Trieste, 63, 35121 Padova PD}
\newcommand{\sitoAzienda}{https://www.math.unipd.it/}
\newcommand{\emailAzienda}{conti@math.unipd.it}
\newcommand{\partitaIVAAzienda}{P.IVA 00742430283}

% Dati del Tutor Interno (Docente)
\newcommand{\titoloTutorInterno}{Prof.}
\newcommand{\nomeTutorInterno}{Davide}
\newcommand{\cognomeTutorInterno}{Bresolin}

\newcommand{\prospettoSettimanale}{
     % Personalizzare indicando in lista, i vari task settimana per settimana
     % sostituire a XX il totale ore della settimana
    \begin{itemize}
        \item \textbf{Prima Settimana - Background (40 ore)} 
        \begin{itemize}
            \item Verifica formale dei flussi: questo è un processo analitico e matematico che mira a garantire la correttezza e la sicurezza dei dati durante il loro attraversamento attraverso sistemi software o reti, identificando potenziali vulnerabilità e errori nel flusso dei dati.
            \item Studio dei dispositivi medici robotici domiciliari: valutazione e l'analisi di robot autonomi progettati per fornire servizi sanitari e assistenza medica direttamente a casa dei pazienti. Questo processo comprende la valutazione delle loro capacità diagnostiche, terapeutiche e di monitoraggio dei pazienti, nonché l'analisi dei dati raccolti e la loro integrazione nei sistemi sanitari esistenti per migliorare la qualità delle cure domiciliari.
            \item Studio dell'architettura dei robot sociali nella letteratura: coinvolge l'analisi dei componenti hardware e software di questi dispositivi, oltre alla loro progettazione e programmazione per garantire l'interazione umana naturale e l'adattamento a diverse situazioni sociali. Questo processo comprende la valutazione della percezione, dell'elaborazione delle informazioni, delle capacità di apprendimento e delle modalità di comunicazione dei robot sociali.
        \end{itemize}
        \item \textbf{Seconda Settimana - Studio di un'Architettura Esistente (40 ore)}
        \begin{itemize}
            \item Studio dell'architettura generale: questo processo coinvolge l'analisi e la progettazione di sistemi di comunicazione robusti e sicuri che consentano al robot di scambiare dati con server esterni per l'elaborazione, il controllo e l'accesso a informazioni aggiornate. Le modalità di implementazione includono la definizione di protocolli di comunicazione, la gestione dei dati trasmessi e la sicurezza delle trasmissioni al fine di garantire un funzionamento affidabile e protetto del robot sociale.
        \end{itemize}
        \item \textbf{Terza Settimana - Studio di un'Architettura Esistente (40 ore)}
        \begin{itemize}
            \item Identificazione dei flussi dati e definizione formale: coinvolge l'analisi e la documentazione dei percorsi attraverso cui i dati sono raccolti, elaborati e scambiati all'interno del sistema. Questo processo aiuta a comprendere come l'architettura gestisce le informazioni e può essere utile per identificare potenziali miglioramenti o problemi di sicurezza nei flussi di dati esistenti.
        \end{itemize}
        \item \textbf{Quarta Settimana - Progettazione dell'Architettura Sicura Generica (40 ore)} \\ Definizione di un insieme di principi e misure di sicurezza per proteggere il robot da minacce esterne e garantire la privacy dei dati utente. Ciò include la crittografia dei dati, l'accesso controllato, la gestione delle vulnerabilità e l'aggiornamento regolare del software per prevenire potenziali rischi di sicurezza.
        \item \textbf{Quinta Settimana - Progettazione dell'Architettura Sicura Generica (40 ore)} \\ Definizione di un insieme di principi e misure di sicurezza per proteggere il robot da minacce esterne e garantire la privacy dei dati utente. Ciò include la crittografia dei dati, l'accesso controllato, la gestione delle vulnerabilità e l'aggiornamento regolare del software per prevenire potenziali rischi di sicurezza.
        \item \textbf{Sesta Settimana - Progettazione dell'Architettura Sicura Generica e Adattamento all'Architettura Preesistente (40 ore)} \\ Rielaborazione e l'implementazione dei concetti e delle pratiche di sicurezza descritte in precedenza. Questo processo richiede l'integrazione delle misure di protezione, come la crittografia e i controlli di accesso, nell'infrastruttura esistente, al fine di assicurare che il social robot sia in grado di operare in modo più sicuro e resiliente contro potenziali minacce esterne.
        \item \textbf{Implementazione Generale dell'Architettura Sicura (40 ore)} \\ Rielaborazione e l'implementazione dei concetti e delle pratiche di sicurezza descritte in precedenza. Questo processo richiede l'integrazione delle misure di protezione, come la crittografia e i controlli di accesso, nell'infrastruttura esistente, al fine di assicurare che il social robot sia in grado di operare in modo più sicuro e resiliente contro potenziali minacce esterne.
        \item \textbf{Ottava Settimana - Verifica e Validazione (40 ore)} \\  L'analisi e il test dell'architettura per garantire che le misure di sicurezza proposte siano efficaci e che l'architettura stessa funzioni correttamente. Ciò può includere simulazioni, test di penetrazione, e verifiche di conformità alle specifiche di sicurezza per assicurarsi che il sistema sia robusto contro minacce e vulnerabilità.
    \end{itemize}
    
}

% Indicare il totale complessivo (deve essere compreso tra le 300 e le 320 ore)
\newcommand{\totaleOre}{320}

\newcommand{\obiettiviObbligatori}{
	 \item \underline{\textit{O01}}: primo obiettivo;
	 \item \underline{\textit{O02}}: secondo obiettivo;
	 \item \underline{\textit{O03}}: terzo obiettivo;
	 
}

\newcommand{\obiettiviDesiderabili}{
	 \item \underline{\textit{D01}}: primo obiettivo;
	 \item \underline{\textit{D02}}: secondo obiettivo;
}

\newcommand{\obiettiviFacoltativi}{
	 \item \underline{\textit{F01}}: primo obiettivo;
	 \item \underline{\textit{F02}}: secondo obiettivo;
	 \item \underline{\textit{F03}}: terzo obiettivo;
}