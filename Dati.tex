%----------------------------------------------------------------------------------------
%   USEFUL COMMANDS
%----------------------------------------------------------------------------------------

\newcommand{\dipartimento}{Dipartimento di Matematica ``Tullio Levi-Civita''}

%----------------------------------------------------------------------------------------
% 	USER DATA
%----------------------------------------------------------------------------------------

% Data di approvazione del piano da parte del tutor interno; nel formato GG Mese AAAA
% compilare inserendo al posto di GG 2 cifre per il giorno, e al posto di 
% AAAA 4 cifre per l'anno
\newcommand{\dataApprovazione}{Data}

% Dati dello Studente
\newcommand{\nomeStudente}{Matteo}
\newcommand{\cognomeStudente}{Soldà}
\newcommand{\matricolaStudente}{1226319}
\newcommand{\emailStudente}{matteo.solda.1@studenti.unipd.it}
\newcommand{\telStudente}{+ 39 342 029 3341}

% Dati del Tutor Aziendale
\newcommand{\nomeTutorAziendale}{Prof. Mauro}
\newcommand{\cognomeTutorAziendale}{Conti}
\newcommand{\emailTutorAziendale}{conti@math.unipd.it}
\newcommand{\telTutorAziendale}{+39 049 827 1488}
\newcommand{\ruoloTutorAziendale}{Prof.}

% Dati dell'Azienda
\newcommand{\ragioneSocAzienda}{Università degli Studi di Padova}
\newcommand{\indirizzoAzienda}{Dipartimento di Matematica - Via Trieste, 63, 35121 Padova PD}
\newcommand{\sitoAzienda}{https://www.math.unipd.it/}
\newcommand{\emailAzienda}{conti@math.unipd.it}
\newcommand{\partitaIVAAzienda}{P.IVA 00742430283}

% Dati del Tutor Interno (Docente)
\newcommand{\titoloTutorInterno}{Prof.}
\newcommand{\nomeTutorInterno}{}
\newcommand{\cognomeTutorInterno}{}

\newcommand{\prospettoSettimanale}{
     % Personalizzare indicando in lista, i vari task settimana per settimana
     % sostituire a XX il totale ore della settimana
    \begin{itemize}
        \item \textbf{Prima Settimana - Analisi dello Stato dell'Arte dei Social Robot (XX ore)}
        \begin{itemize}
            \item 
        \end{itemize}
        \item \textbf{Seconda Settimana - Sottotitolo (XX ore)} 
        \begin{itemize}
            \item ;
        \end{itemize}
        \item \textbf{Terza Settimana - Sottotitolo (XX ore)} 
        \begin{itemize}
            \item ;
        \end{itemize}
        \item \textbf{Quarta Settimana - Sottotitolo (XX ore)} 
        \begin{itemize}
            \item ;
        \end{itemize}
        \item \textbf{Quinta Settimana - Sottotitolo (XX ore)} 
        \begin{itemize}
            \item ;
        \end{itemize}
        \item \textbf{Sesta Settimana - Sottotitolo (XX ore)} 
        \begin{itemize}
            \item ;
        \end{itemize}
        \item \textbf{Settima Settimana - Sottotitolo (XX ore)} 
        \begin{itemize}
            \item ;
        \end{itemize}
        \item \textbf{Ottava Settimana - Conclusione (XX ore)} 
        \begin{itemize}
            \item ;
        \end{itemize}
    \end{itemize}
}

% Indicare il totale complessivo (deve essere compreso tra le 300 e le 320 ore)
\newcommand{\totaleOre}{}

\newcommand{\obiettiviObbligatori}{
	 \item \underline{\textit{O01}}: primo obiettivo;
	 \item \underline{\textit{O02}}: secondo obiettivo;
	 \item \underline{\textit{O03}}: terzo obiettivo;
	 
}

\newcommand{\obiettiviDesiderabili}{
	 \item \underline{\textit{D01}}: primo obiettivo;
	 \item \underline{\textit{D02}}: secondo obiettivo;
}

\newcommand{\obiettiviFacoltativi}{
	 \item \underline{\textit{F01}}: primo obiettivo;
	 \item \underline{\textit{F02}}: secondo obiettivo;
	 \item \underline{\textit{F03}}: terzo obiettivo;
}