%----------------------------------------------------------------------------------------
%	DESCRIPTION OF THE PRODUCTS THAT ARE BEING EXPECTED FROM THE STAGE
%----------------------------------------------------------------------------------------
\section*{Prodotti attesi}
% Personalizzare definendo i prodotti attesi (facoltativo)
Lo studente dovrà produrre una relazione scritta che illustri i seguenti punti.
\begin{enumerate}
    \item \textbf{Analisi dello Stato dell'Arte} \\
    Durante questa prima fase, lo studente avrà l'opportunità di condurre un'analisi completa dello stato dell'arte dei robot sociali, compresa la ricerca approfondita di fonti accademiche e tecniche, l'analisi delle tecnologie chiave impiegate, la valutazione delle loro prestazioni e della loro rispetto riguardo la privatezza dei dati raccolti durante l'utilizzo. L'obiettivo principale sarà quello di acquisire una conoscenza approfondita delle tendenze attuali e delle sfide nel campo dei robot sociali, contribuendo così a una migliore comprensione del settore di riferimento.
    
    \item \textbf{Vulerabilità ed Exploiting} \\
    Lo studente sarà responsabile di condurre una ricerca mirata per identificare potenziali vulnerabilità nei sistemi robotici sociali. Questa ricerca comprenderà l'analisi delle configurazioni software, l'esplorazione delle interfacce di comunicazione e l'identificazione di possibili punti di ingresso per attacchi informatici. Una volta individuate le vulnerabilità potenziali, lo studente procederà a condurre test di penetration testing eticamente approvati, al fine di valutare la reale esposizione ai rischi dei sistemi in esame. L'obiettivo sarà quello di fornire un quadro completo delle potenziali minacce e dei possibili scenari di attacco.
    
    \item \textbf{Mitigazione dei Rischi} \\
    Lo studente si concentrerà sulla progettazione e sperimentazione di una metodologia per la mitigazione dei rischi legati alle vulnerabilità identificate nei robot sociali. Questa metodologia avrà l'obiettivo di sviluppare misure di sicurezza efficaci che possano prevenire, rilevare o mitigare potenziali attacchi informatici. Questo processo comprenderà l'implementazione di contromisure tecniche, come patching del software, configurazioni avanzate di sicurezza e monitoraggio costante del sistema, oltre a promuovere l'educazione e la consapevolezza sulla sicurezza tra gli utilizzatori dei robot sociali. L'obiettivo finale sarà quello di fornire una guida pratica e efficace per affrontare le vulnerabilità e proteggere la sicurezza e la privacy nelle interazioni con i robot sociali.
\end{enumerate}

Nel qual caso in cui lo studente, in seguito all'analisi, abbia ancora tempo a sua disposizione ... .
